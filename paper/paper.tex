
\documentclass[]{bytedance_seed}

% single-column: \documentclass[]{bytedance_seed}, 
%Please prioritize using single-column。

% twocolumn: \documentclass[twocolumn]{bytedance_seed}

\usepackage[toc,page,header]{appendix}


%%%%%%%%%%%%%%%%%%%%%%%%%%%%%%%%%%%%

\usepackage{minitoc}
\usepackage{amssymb}
\usepackage[table]{xcolor}
\usepackage{wrapfig}
\usepackage[utf8]{inputenc}  % for UTF-8 support
\usepackage{float}  % for listing environment
\usepackage{minted}  % for code highlighting
\usepackage{textcomp}  % for special symbols
\usepackage{footmisc}
% table 

\definecolor{papergreen}{HTML}{008000} % 较暗的绿色
\definecolor{paperred}{HTML}{D00000}   % 较暗的红色

% 命令 \goodres{主要值}{变化值}
% 用于显示一个好的结果 (绿色, +)
\newcommand{\goodres}[2]{%
    #1 {\color{papergreen}\small (+#2)}%
}
% 命令 \badres{主要值}{变化值}
% 用于显示一个坏的结果 (红色, -)
\newcommand{\badres}[2]{%
    #1 {\color{paperred}\small (-#2)}%
}
% 用于显示一个中性或无对比的结果
\newcommand{\neutralres}[2]{%
    #1 {\color{gray}\small (#2)}%
}
%%%%%%%%%%%%%%%%%%%%

\newcommand{\bench}{\texttt{Our-Bench}\xspace}
\definecolor{navyblue}{HTML}{0071BC}
\definecolor{oai-green-200}{HTML}{D7EADF} % 浅绿色
\definecolor{oai-green-400}{HTML}{A6D3B8} % 中等绿色
\definecolor{oai-green-600}{HTML}{63AC86} % 深绿色
\definecolor{oai-gray-300}{HTML}{D9D9D9}  % 浅灰色
\definecolor{oai-gray-600}{HTML}{808080}  % 深灰色
\definecolor{L1_Purple}{HTML}{E6E0F8} % A light purple for L1
\definecolor{L2_Orange}{HTML}{FCEFD8} % A light orange/yellow for L2
\definecolor{L3_Green}{HTML}{EFF8E0}  % A light green for L3
\definecolor{L4_Blue}{HTML}{E0F0F8}   % A light blue for L4
%%%%%%%%%%%%%%%%%%%%

% \title{SpatialTree: How Spatial Abilities Branch Out in MLLMs}
\title{SpatialTree: How Spatial Abilities Branch Out in MLLMs}

\author[\blacktriangle,\bigstar,*]{Yuxi Xiao}
\author[\diamondsuit,\bigstar,*]{Longfei Li}
\author[\bigstar]{Shen Yan}
\author[\bigstar]{Xinhang Liu}
\author[\blacktriangle]{Sida Peng}
\authorbreak
\author[\diamondsuit]{Yunchao Wei}
\author[\blacktriangle]{Xiaowei Zhou}
\author[\bigstar, \dagger]{Bingyi Kang}

%论文单位请使用ByteDance Seed
\affiliation[\blacktriangle]{Zhejiang University}
\affiliation[\bigstar]{ByteDance Seed}
\affiliation[\diamondsuit]{Beijing Jiaotong University}


\contribution[\dagger]{Project Lead}
\contribution[*]{Equal Contribution}

\abstract{
Spatial abilities in multimodal LLMs manifest hierarchically, spanning multiple levels from perception, through reasoning, to agentic tasks in 3D environments. However, cross-level dependencies among these abilities and their potential to guide data and training strategies for systematically scaling up remain largely unexplored. To address this, we propose SpatialTree, a hierarchical taxonomy organizing spatial abilities from low-level perception (L1) through mental mapping (L2) and mental simulation (L3) to agentic competence (L4). Building on this taxonomy, we construct the first hierarchical, capability-centric benchmark using our Spatial Engine. Guided by correlation analyses on our benchmark, we perform supervised fine-tuning (SFT) and reinforcement learning (RL) at different ability levels to probe their inter-dependencies. We find that low-level abilities (L1) are largely independent but can transfer to higher-level abilities and exhibit synergy during SFT. Furthermore, reinforcement learning proves more effective when applied to higher-level tasks rather than low-level abilities (L1), resulting in improved overall spatial performance. We hope our taxonomy and findings will inspire further research to systematically branch out spatial abilities in MLLMs through dependency-aware training.
}

\date{\today}
\correspondence{Bingyi Kang}
% You can add additional info fields as follows 
\checkdata[Project Page]{\url{xxx}}

\begin{document}
\maketitle

%不需要目录就注释掉 注意目录不要和第一页放在一块 要有\newpage
%\newpage
%\tableofcontents
%\newpage

\begin{figure}[!ht]
\centering
\includegraphics[width=0.98\linewidth]{figures/teaser_tree.pdf}
\caption{\textbf{SpatialTree.} Inspired by cognitive science, our proposed SpatialTree organizes spatial intelligence into a four-layer hierarchy (L1-L4). Rooted in foundational multi-modal capabilities (L0), the tree progressively branches from Basic perception (L1) to agentic competence (L4).}
\label{fig:spatree}
\end{figure}

\footnotetext[1]{Work done during Yuxi Xiao and Longfei Li’s internship at Bytedance Seed.}

\section{Introduction}
\label{sec:intro}

Developing Spatial Intelligence (SI) Systems to perceive, reason, and interact within the physical world is a long-standing challenge across cognitive science~\citep{tolman1948cognitive, shepard1988mental, newcombe2000making}, symbolic AI~\citep{kuipers1978modeling, kuipers2000spatial, harnad1990symbol}, and robotics~\citep{durrant2006simultaneous, thrun2002probabilistic}. However, progress has historically been limited by the lack of a unified model capable of integrating perception, reasoning, and action. The emergence of Multimodal Large Language Models (MLLMs), with their powerful vision-language understanding and reasoning capabilities, has opened new opportunities for advancing SI. 

% --- Full Paragraph based on the "Task Evolve" Roadmap ---
Recent research on spatial intelligence (SI) in MLLMs has largely followed a task-centric trajectory. Early works focused on simple spatial tasks in single images~\citep{ma20243dsrbench, wang2024spatial,fu2024blink}, such as relative object positioning and size estimation. Later studies expanded these tasks to 3D grounding, detection, and captioning from point clouds~\citep{zhu2024llava,hong20233d,xu2024pointllm}. With multi-view and video-capable VLMs, benchmarks quickly diversified~\citep{yang2025thinking, wang2025site, wang2025spatialviz, gholami2025spatial, yang2025mmsi, jia2025omnispatial}, covering a wide array of tasks from spatiotemporal reasoning to egocentric and dynamic object understanding.

However, existing task-centric benchmarks, while useful, remain fragmented—often focusing on isolated or overlapping spatial tasks. This makes it difficult to understand not only an MLLM’s overall spatial intelligence, but also how these abilities emerge and transfer across levels. This motivates us to ask:

\begin{quote}
Can we move beyond fragmented, task-centric evaluations to uncover a compact set of atomic capabilities that reveal how spatial abilities emerge, interact, and transfer?
\end{quote}

Inspired by Piaget’s theory in cognitive science~\cite{piaget2013construction}, we advocate a \emph{capability-centric} paradigm for spatial intelligence (SI). We further decompose SI into a multi-level capability tree (Fig.~\ref{fig:spatree}). Based on this taxonomy, we construct the first comprehensive benchmark for SI in MLLMs, offering comprehensive ability coverage and a diverse set of evaluation metrics beyond simple multiple-choice tests used in prior works. We also develop a Spatial Engine, an extendable annotation framework. It integrates multiple perception models to generate annotations for each capability layer. At the highest level (L4), we propose a spatial action mapping which converts continuous actions into discrete, high-level motion primitives, providing MLLMs with an executable action space for agentic tasks. We leverage the proposed Spatial Engine to generate diverse annotation data from public datasets covering video games, robot manipulation, and human-object interactions, with details provided in Sec.~\ref{sec:l4-agentic-competence}. To cover the lower levels (L1–L3), we extract relevant portions from multiple public datasets and benchmarks~\citep{yang2025mmsi,yang2025thinking,jia2025omnispatial,lin2025towards, wang2025spatialviz, wang2024spatial, zhu2024llava, yin2025spatial, xu2025multi, liu2025ir3d}, reorganizing them onto our capability tree. We further enrich questions and evaluation protocols to improve coverage and cross-layer overlap. To address missing capabilities, we introduce \textbf{SpatialPlus}, generated by \textbf{SpatialEngine}, encompassing Orientation (L1), Memory Retrieval (L2), Relational Reasoning (L3), and Agentic Competence (L4). All resulting data and annotations are systematically organized and re-weighted within the SpatialTree benchmark to ensure balanced evaluation across layers.

% ** 最后一段实验+结论 等搞完之后再写 **
Evaluation on SpatialTree-Bench reveals a clear hierarchical structure in spatial intelligence: low-level abilities (L1–L2) are largely independent, whereas higher-level abilities (L3–L4) exhibit strong interdependencies, reflecting their compositional nature. To systematically validate these correlations, we conduct targeted supervised fine-tuning (SFT) and prompting experiments. Our SFT results reveal a clear cross-level transfer, where training on a single L1 ability boosts high-level L4 tasks despite harming same-level performance. Crucially, we also find a powerful multi-ability synergy: jointly training these foundational L1 abilities unlocks robust holistic gains, far exceeding the negligible or even negative results of individual SFT. We further confirm this dependency via prompting, as explicitly grounding L4 agentic tasks with L1 perceptual cues significantly enhances performance.

In summary, our key contributions are:
\begin{itemize}
    \item We propose the first capability-centric benchmark for spatial intelligence, providing a systematic framework for benchmarking and interpretable analysis.
    \item We design a series of experiments to investigate how abilities emerge and transfer across both same-level and cross-level tasks, revealing the hierarchical relationships from lower- to higher-level capabilities.
\end{itemize}
%\subsection{Hello World}
\section{Related Work}
\label{sec:related_work}

\paragraph{Spatial Cognitive Modeling.} 
Understanding spatial cognition has long been a central goal in cognitive science and AI. A common insight from classical theories is that spatial abilities are hierarchical, ranging from basic perception and sensorimotor interactions to higher-level reasoning and planning. Piaget~\cite{piaget2013construction} highlighted the developmental progression of such abilities, Tolman~\cite{tolman1948cognitive} introduced the idea of cognitive maps to represent environments for flexible navigation, and Kuipers~\cite{kuipers1978modeling, kuipers2000spatial} formalized a hierarchical spatial representation linking local perception to global knowledge. More recent symbolic and neural approaches~\cite{shepard1988mental, newcombe2000making} extend these insights to computational models of spatial representation, memory, and reasoning. These studies collectively motivate our \emph{SpatialTree}, which organizes spatial intelligence into multi-level capabilities, bridging classical theory with systematic computational evaluation.


\paragraph{Multi-modal Large Language Models.} 
The success of GPT-3~\cite{brown2020language} and GPT-3.5~\cite{openai-gpt-3.5} demonstrated the potential of large language models for complex linguistic understanding and reasoning. GPT-4V~\cite{openai2023gpt4v} extends GPT-4~\cite{achiam2023gpt} with visual inputs, enabling single-image understanding and basic spatial reasoning. Open-sourced models such as LLaVA~\cite{liu2023visual} and QwenVL~\cite{bai2023qwen} gradually added multi-image and video capabilities, supporting spatiotemporal reasoning. Reasoning-augmented LLMs, pioneered by OpenAI O1~\cite{jaech2024openai} and DeepSeek-R1~\cite{guo2025deepseek}, integrate chain-of-thought and reinforcement learning to enhance high-level inference. Building on these advances, GPT-4O~\cite{hurst2024gpt} and Gemini 2.5~\cite{comanici2025gemini} combine perception and reasoning to support complex, agentic decision-making. Collectively, these milestones progressively enable hierarchical spatial intelligence in MLLMs, motivating structured benchmarks and evaluation frameworks across low-level perception, intermediate reasoning, and high-level agentic competence.



\paragraph{Benchmarks for Spatial Intelligence in MLLMs.} 
Benchmarks for spatial abilities in MLLMs have evolved alongside the models themselves. Early efforts, such as BLINK~\cite{fu2024blink}, SpatialEval~\cite{wang2024spatial}, and 3DSR-Bench~\cite{ma20243dsrbench}, focused on evaluating spatial understanding tasks in single images, including distance estimation, relational question answering, and spatial captions. As MLLMs increasingly support multi-frame and video inputs, benchmarks such as VSI-Bench~\cite{yang2025thinking} and MMSI-Bench~\cite{yang2025mmsi} have emerged to evaluate spatial reasoning across multiple views and dynamic scenes. To further enrich task diversity and coverage, Omnispatial~\cite{jia2025omnispatial}, SITE~\cite{wang2025site}, and IR3D-Bench~\cite{liu2025ir3d} extend benchmarks to geometry puzzles, dynamic reasoning, and inverse rendering tasks. Built upon prior efforts, our SpatialTree benchmark systematically organizes spatial abilities into a hierarchical framework, providing the first thorough evaluation across different capabilities.
\section{The SpatialTree Framework}
\label{sec:spatialtree_framework}

\begin{figure*}
    \centering
    \includegraphics[width=0.98\linewidth]{figures/Samples.pdf}
    \caption{\textbf{A gallery of representative tasks.} The benchmark, structured by the SpatialTree framework, covers a diverse set of spatial capabilities spanning the four-level hierarchy from perception(L1) to agentic competence(L4).}
    \vspace{-1.5em}
    \label{fig:samples}
\end{figure*}

In this section, we present \textbf{SpatialTree}, a top-down hierarchical decomposition of spatial capabilities into four levels, from high-level agentic competence (L4) to foundational perception (L1). Different samples for different level of capabilities are shown in Fig.~\ref{fig:samples}.

\subsection{\textbf{Agentic Competence}}
\label{sec:l4-agentic-competence}
We begin from the ultimate objective of a \textit{Spatial AI Agent} --- an MLLM-driven system that integrates multi-modal observations, updates its memory, and selects actions to interact with the 3D world in an intuitive manner. Formally, the agent performs sequential decision-making by modeling:
% \begin{equation}
% (S_t, A_t, M_t) \sim P_\theta \Big( \cdot \;\big|\; O_t, H_{t-1} \Big), 
% where 
% H_{t-1} = \left\{ (O_0, A_0, M_0), \dots, (O_{t-1}, A_{t-1}, M_{t-1}) \right\}
% \label{eq:agent_model}
% \end{equation}
\begin{align}
(S_t, A_t, M_t) & \sim P_\theta \Big( \cdot \;\big|\; O_t, H_{t-1} \Big), \label{eq:agent_model} \\
\text{where} \quad H_{t-1} & = \left\{ (O_0, A_0, M_0), \dots, (O_{t-1}, A_{t-1}, M_{t-1}) \right\} \nonumber
\end{align}
\vspace{-0.2em}
where $O_t \in \mathcal{O}$ is the current multi-modal observation, $S_t \in \mathcal{S}$ the internal latent state (e.g., goal, plan, or belief), $A_t \in \mathcal{A}$ the chosen action, and $M_t \in \mathcal{M}$ the updated memory representation. MLLMs are expected to output interactive actions executable across 3D environments and embodiments, such as games, simulators, and the physical world. Unlike Vision-Language Action Models (VLAs) decording the low-level control signals in robotics~\citep{intelligence2025pi_}, MLLMs take the language as the only interface to link with environments like GUI Agents~\citep{UI-TARS}.  

\textbf{Spatial Action Mapping.} % In your preamble, make sure you have: \usepackage{booktabs} \usepackage{tabularx}
In the context of spatial agents, navigation and manipulation represent the most common forms of interaction within 3D environments. We address each with a distinct action space design. For navigation, we conceptualize agent movement as a series of camera motion controls (referring to recent video world models~\citep{genie3,yume}). To enable precise and intuitive control, we decompose complex camera movements into a set of fundamental motion primitives inspired by established cinematography techniques. This approach allows us to translate high-level language instructions (e.g., "move to the left," "look up") into a structured, low-level action space. The six fundamental primitives, their corresponding cinematic terms, degrees of freedom (DoF), and parameterization are detailed in Tab.~\ref{tab:spatial_action_mapping}.
\begin{table}[t!]
\centering
\caption{\textbf{Spatial Action Mapping.} This table defines a standardized interface that maps continuous 6-DoF motions and discrete control signals into action primitives with unified parameterization, enabling MLLMs to plan and execute embodied behaviors for agentic competence evaluation.}
\label{tab:spatial_action_mapping}
\vspace{-0.5em}
\small
\resizebox{\columnwidth}{!}{%
\begin{tabular}{@{}lll>{\centering\arraybackslash}m{3.0cm}>{\centering\arraybackslash}m{2.2cm}cc@{}} 
\toprule
\textbf{Primitive} & \textbf{Primitive Term} & \textbf{Category} & \textbf{Description} & \textbf{Action Mapping} & \textbf{Param.} & \textbf{Threshold} \\
\midrule
$P_{\text{truck}}$ & Truck & Translation & Move camera left/right (X-axis) & $A / D$ & $v_x$ & $\pm 0.01$ m/s \\
$P_{\text{dolly}}$ & Dolly & Translation & Move camera forward/backward (Z-axis) & $W / S$ & $v_z$ & $\pm 0.01$ m/s \\
$P_{\text{pedestal}}$ & Pedestal & Translation & Move camera up/down (Y-axis) & $Q / E$ & $v_y$ & $\pm 0.01$ m/s \\
$P_{\text{pan}}$ & Pan & Rotation & Turn camera left/right (yaw) & $\leftarrow / \rightarrow$ & $\omega_y$ & $\pm 0.5^\circ$/s \\
$P_{\text{tilt}}$ & Tilt & Rotation & Tilt camera up/down (pitch) & $\uparrow / \downarrow$ & $\omega_x$ & $\pm 0.5^\circ$/s \\
$P_{\text{roll}}$ & Roll & Rotation & Roll camera CW/CCW (roll) & $Z / X$ & $\omega_z$ & $\pm 0.5^\circ$/s \\
\midrule
$O_{\text{gripper}}$ & Gripper & Gripper Control & Open or close the gripper & $G / H$ & State $\in \{0,1\}$ & N/A \\
$O_{\text{push/pull}}$ & Push/Pull & Gesture & Push or pull object along forward axis & $P / L$ & Dir. $\in \{-1,+1\}$ & N/A \\
$O_{\text{grab}}$ & Grab & Gesture & Grab or release object & None & State $\in \{0,1\}$ & $G / H$ \\
\bottomrule
\end{tabular}}
\vspace{-5mm}
\end{table}

Formally, the camera trajectories are defined with a series of 
Camera-to-World (C2W) transformation matrices
$\mathcal{T}_{\text{motion}} = \{\mathbf{T}|_{0}^{i}, i=0,1,\dots,t\}$,
while the camera transformation at each moment is 
$\mathcal{T}_{i \rightarrow {i+1}} = \textbf{T}_{i+1}\textbf{T}_{i}^{-1}, 
\; i=0,1,\dots,t-1$. Then the continuous camera transformation can be decomposed into different components corresponding to each motion primitive, and discretized into the navigation action $A_{\text{nav}}$ using a speed threshold: 
% \begin{equation}
% \mathbf{A}_i^{\text{nav}} 
% = \mathbf{T}_{i \rightarrow i+1} 
% = \{\Delta \mathbf{R}, \Delta \mathbf{t}\} 
% \approx 
% \{ t_i \cdot v_i, \; t_k \cdot\omega_k \mid i,k \in \{x,y,z\}, \; t_i, t_k \in \mathbb{Z}_{\ge 0} \},
% \label{eq:discrete_navigation_action}
% \end{equation}
\begin{align}
\label{eq:discrete_navigation_action}
\mathbf{A}_i^{\text{nav}} 
= \mathbf{T}_{i \rightarrow i+1} 
& = \{\Delta \mathbf{R}, \Delta \mathbf{t}\} \\
& \approx 
\left\{ \begin{array}{l}
t_i \cdot v_i, \; t_k \cdot\omega_k \mid i,k \in \{x,y,z\}, \\
t_i, t_k \in \mathbb{Z}_{\ge 0} 
\end{array} \right\} \nonumber
\end{align}
where $\Delta \mathbf{R} = (\Delta R_x, \Delta R_y, \Delta R_z)$ represents the rotation components obtained via Euler decomposition, $\Delta \mathbf{t} = (\Delta t_x, \Delta t_y, \Delta t_z)$ denotes the translation components along the $x$, $y$, and $z$ axes, and $t_i, t_k$ are discrete integers ranging from 0 up to the video frame rate (FPS). For manipulation, we focus on two representative scenarios to simplify the problem and enable controlled evaluation: human-hand manipulation and robotic gripper manipulation. For the gripper setting, we include gripper open/close actions along with wrist-level 6-DoF motion. For the human-hand setting, we define a small set of intuitive gesture primitives (i.e., push, pull, grab) seen in Tab.~\ref{tab:spatial_action_mapping} that capture essential interaction patterns. These manually defined mappings create a unified yet tractable action space for analyzing MLLMs' planning and manipulation competence.

Building on the proposed spatial action mapping, we curate annotated data from diverse sources, including human-hand manipulation videos, navigation video games, robotic arm manipulation datasets, and simulation environments. This unified dataset enables us to evaluate whether MLLMs can accurately plan and execute actions in the defined metric action space. Further implementation details are provided in Sec.~\ref{sec:si-engine} and in the experimental section.


\subsection{\textbf{Mental Simulation}}
\label{sec:l3-mental-simulation}
Reasoning and planning prior to action execution are essential components of Multimodal Large Language Models (MLLMs), aligning naturally with the Chain-of-Thought paradigm in language model reasoning. In spatial cognitive science, this process is commonly referred to as mental simulation. We further decompose mental simulation into two core components: causal reasoning and sequential planning.

% \textbf{Causal Reasoning} enables MLLMs to model spatial interactions, physical dynamics, and entity relationships within a simulated mental space. It encompasses tasks such as reasoning about object geometry (e.g., how shapes interlock in spatial puzzles), predicting dynamic motion (e.g., how an object traverses a path under kinematic constraints), and analyzing semantic–spatial relations (e.g., object A is to the left of object B). Through such reasoning, MLLMs mentally simulate cause–effect chains in spatial scenarios, thereby establishing the logical substrate upon which subsequent planning can be carried out.

\textbf{Causal Reasoning} allows MLLMs to model spatial interactions, physical dynamics, and entity relationships within a simulated mental space. It includes reasoning about object geometry (e.g., how shapes interlock in spatial puzzles), motion prediction (e.g., how an object traverses a path), and analyzing semantic–spatial relations (e.g., object A is left of object B). By mentally simulating cause–effect chains in spatial scenarios, MLLMs establish the logical substrate for subsequent planning.

% \textbf{Sequential Planning} translates causal insights into coherent, goal-directed action plans formulated within the language space. This process involves designing high-level, step-by-step operational strategies (e.g., "first, move towards the door, then turn right, and finally interact with the handle") and generating abstract navigational routes that respect spatial logic (e.g., "go around the table to reach the sofa"). By chaining these linguistic action primitives, MLLMs formulate strategic plans, ensuring the conceptual sequence aligns with the overarching goal before any low-level execution is attempted.

\textbf{Sequential Planning} converts causal insights into coherent, goal-directed action plans expressed in language. It entails designing high-level, step-by-step strategies (e.g., "first move toward the door, then turn right, and finally interact with the handle") and generating abstract routes that respect spatial logic (e.g., "go around the table to reach the sofa"). By chaining linguistic action primitives, MLLMs produce strategic plans that ensure the conceptual sequence aligns with the overarching goal before any low-level execution.




\vspace{-0.7em}
\subsection{\textbf{Mental Mapping}}
\label{sec:l2-mental-mapping}
\vspace{-0.7em}
Level 3's advanced mental simulation requires a coherent internal world model, a foundation provided by Level 2's mental mapping. This process constructs a dynamic 3D representation of the environment by relying on two essential facets.
The first is spatial understanding: the ability to interpret the immediate scene. This includes recognizing objects and their affordances, mapping the spatial relations between them, and understanding the scene from various perspectives (perspective taking). In essence, it's about making sense of what is currently perceived.
The second facet is memory. It allows the agent to retain this understanding over time, retrieving past observations to build a cognitive map that extends far beyond the current field of view. This creates a persistent and comprehensive mental model of the world. 
Ultimately, these two facets—understanding the present and remembering the past—organize and integrate the foundational perceptual data from L1, paving the way for L3's predictive simulations.


\vspace{-0.7em}
\subsection{\textbf{Perception}}
\label{sec:l1-perception}
\vspace{-0.7em}
Perception forms the foundation for high-level spatial reasoning. We categorize L1 Perception into five core aspects:
\textbf{Orientation}: Captures spatial alignment, crucial for understanding the agent’s pose and maintaining balance. Key sub-tasks are \textit{Gravity} (estimating pitch and roll to determine “up”/“down”) and \textit{Object Orientation} (recognizing object poses), supporting scene reconstruction and manipulation.
\textbf{Geometry}: Involves spatial form, size, and metric relationships. Sub-tasks include \textit{Size}, \textit{Shape}, and \textit{Distance}, enabling reasoning about object properties and facilitating navigation and grasping.
\textbf{Motion}: Encodes spatial dynamics over time. Sub-tasks are \textit{Egocentric Motion} (self-motion estimation) and \textit{Allocentric Motion} (tracking object or scene changes), critical for predicting future states and planning actions.
\textbf{Relation}: Concerns spatial relationships between entities. Sub-tasks include \textit{Correspondence} (matching entities across views) and \textit{Relative Direction} (e.g., left of, in front of), supporting object tracking, path planning, and interaction reasoning.
\textbf{Localization}: Anchors perception within 3D space. Sub-tasks include \textit{3D Detection} (identifying object extents) and \textit{3D Grounding} (associating observations with coordinates), enabling scene reconstruction, navigation, and embodied reasoning.

\section{Spatial Engine: The Annotation Pipeline}
\label{sec:si-engine}
We propose an extensible data engine designed to generate annotations for every layer of the SpatialTree. Our approach begins with a diverse set of low-level 3D perception models, each tailored to a specific task, including metric depth estimation~\citep{wang2025moge2,yang2024depth}, orientation estimation~\citep{orient_anything}, gravity estimation~\citep{veicht2024geocalib}, correspondence matching~\citep{leroy2024grounding, xiangli2025doppelgangers++}, 3D localization~\citep{mao2025spatiallm}, 3D point tracking~\citep{xiao2024spatialtracker, xiao2025spatialtrackerv2}, and camera pose estimation~\citep{wang2025vggt, wang2025pi}. Nevertheless, all these comprehensive 3D perception models can be seamlessly encompassed within our taxonomy of five perception abilities.

\textbf{Data Annotation Framework.}
As shown in Fig.~\ref{fig:Data_Engine_Method}, our pipeline encapsulates three hierarchical entities: models, pipelines, and workflows. The lowest level comprises the perception models described above, along with advanced MLLMs for semantic captioning. Building upon these models, we construct several pipelines that serve as atomic components for higher-level workflows. Specifically, we implement 12 pipelines, including metric 3D reconstruction, 3D orientation alignment, 3D point tracking, and affordance pointing. Each pipeline processes raw sensory data, such as RGB images or 3D point clouds, and produces intermediate outputs that are further integrated by the workflows. Based on these pipelines, we assemble 24 workflows, each targeting a specific combination of abilities, to generate comprehensive annotations for our SpatialTree. These reusable pipelines streamline annotation and facilitate future extensions. Overall, this hierarchical design ensures modularity, scalability, and a clear separation of responsibilities across models, pipelines, and workflows.

\begin{figure}[tbp]
\centering
\includegraphics[width=0.98\linewidth]{figures/SI-Engine.pdf}
\caption{
\textbf{SpatialTree Data Engine.} A highly modular and scalable framework that decomposes high-level spatial tasks into low-level components, supporting human-in-the-loop supervision.
}%
\vspace{-1.5em}
\label{fig:Data_Engine_Method}
\end{figure}

\textbf{Data Resources.}
As seen in the Appendix, our SpatialTree-Bench is constructed by systematically reorganizing numerous recent datasets (detailed in the Appendix)~\citep{yang2025thinking, yang2025mmsi, zhu2024llava, wang2024spatial, yin2025spatial, lin2025towards, jia2025omnispatial, yang2025embodiedbench, wang2025spatialviz, xu2025multi, ma20243dsrbench} to address their scattered capability coverage and over-reliance on simple questions. We first map each question to our SpatialTree framework and then enhance the evaluation protocol; for instance, complex reasoning tasks from CameraBench and MMSI-Bench are converted to a hybrid \textit{multi-option + LLM-as-a-Judge} format for a finer-grained assessment. To fill the remaining capability gaps, we introduce \textbf{SpatialPlus}, a new dataset targeting underrepresented abilities (e.g., L1 Orientation, L1 Shape, L2 Spatial Caption) with a primary emphasis on L4 Agentic Competence. We leverage our proprietary \textbf{SpatialEngine} to automatically create annotations from a diverse array of video sources, including 3D reconstruction datasets, in-game footage~\cite{ju2024miradata}, egocentric manipulation videos~\citep{egodex}, and robotics data~\citep{khazatsky2024droid}. More implementation details are discussed in Appendix.

% Furthermore, to address the remaining gaps in capability coverage, we introduce our \textbf{SpatialPlus} dataset. It is specifically designed to target underrepresented abilities such as Orientation (L1), Shape (L1), and Spatial Caption (L2), with a primary emphasis on the complex tasks of Agentic Competence (L4). 










\section{Evaluation on SpatialTree-Bench}
\label{sec:evaluation_benchmark}

\subsection{Models and Metrics}
\label{sec:models_and_metrics}
\textbf{Benchmarked Models.}
We categorize the evaluated MLLMs into three groups:
(1) Thinking Models, i.e., models augmented with explicit reasoning or chain-of-thought generation mechanisms (reasoning-augmented), including Gemini 2.5 Flash, Gemini 2.5 Pro~\citep{comanici2025gemini}, GLM-4.5V~\citep{hong2025glm}, and SeedVL1.6-Vision~\citep{guo2025seed1};
(2) Non-Thinking Models, which do not explicitly optimize for reasoning-style generation, such as Gemini 2.5-Pro-Nonthink, Gemini 2.5-Flash-Nonthinking, and GPT-4o~\citep{hurst2024gpt}; and
(3) Open-Source Models, including Qwen2.5-VL~\citep{bai2025qwen2}, Qwen3-VL~\citep{yang2025qwen3}, and Kimi-VL~\citep{team2025kimi}, representing recent community-driven multimodal advances. This diverse selection enables comprehensive comparisons across reasoning and non-reasoning paradigms, proprietary and open-source ecosystems, and model scales ranging from 32B to 72B parameters—providing a holistic overview of the current MLLM landscape. A full list of evaluated models is shown in Tab.~\ref{tab:main_table}.

\textbf{Evaluation Metrics.}
Our evaluation employs a multi-faceted set of metrics tailored to the specific abilities at each level of the SpatialTree. For perception and understanding tasks (L1-L2), we primarily use accuracy-based metrics, such as classification accuracy for object recognition, Mean Squared Error (MSE) for distance estimation, and angular difference for orientation tasks. For higher-level reasoning and planning tasks (L3-L4), we measure task success rates. In the case of agentic tasks (L4), we further analyze the quality of generated actions using metrics like positional error (L2 distance) and orientation error (angular difference) against ground-truth trajectories.
\subsection{Overall Performance}
\label{sec:spatialtree_bench}
\begin{table*}[ht!]
    \vspace{-0.4cm}
    \centering
    \resizebox{\linewidth}{!}{
    \begin{tabular}{l|cc|ccccc|cc|cc|cc}
    
    \hline % <--- THIS IS THE NEW "CAP" LINE TO CLOSE THE TOP OF THE TABLE
    
    % --- Header Definition with new colors ---
    \rowcolor{white} 
    & & & 
    \multicolumn{5}{c|}{\cellcolor{L1_Purple} L1 Perception} &  
    \multicolumn{2}{c|}{\cellcolor{L2_Orange} L2 Mental Mapping} &
    \multicolumn{2}{c|}{\cellcolor{L3_Green} L3 Mental Simulation} &
    \multicolumn{2}{c}{\cellcolor{L4_Blue} L4 Agentic Competence} \\
    
    \rowcolor{white} 
    \textbf{Methods} & \textbf{Rank} & \textbf{Avg.} & 
    \textbf{Geom.} & \textbf{Motion} & \textbf{Rel.} & \textbf{Local.} & \textbf{Orient.} &
    \textbf{Underst.} & \textbf{Memory} &
    \textbf{Caus. Reas.} & \textbf{Seq. Plan.} &
    \textbf{Goal Exec.} & \textbf{Open Expl.} \\
    
    \rowcolor{white}
    & & & 
    \small\textcolor{gray}{[0.40]} & \small\textcolor{gray}{[0.15]} & \small\textcolor{gray}{[0.15]} & \small\textcolor{gray}{[0.20]} & \small\textcolor{gray}{[0.10]} &
    \small\textcolor{gray}{[0.70]} & \small\textcolor{gray}{[0.30]} &
    \small\textcolor{gray}{[0.65]} & \small\textcolor{gray}{[0.35]} &
    \small\textcolor{gray}{[0.50]} & \small\textcolor{gray}{[0.50]} \\
    % \hline

    % % --- Baselines ---
    % \rowcolor{navyblue!5}
    % \multicolumn{15}{l}{\textcolor{black}{\textit{Baseline}}} \\
    % Chance Level (Random) & - & - & -- & -- & -- & -- & -- & -- & -- & 25.0 & 36.1 & 28.3 & 25.0 & -- \\
    % Chance Level (Frequency) & - & 34.0 & 62.1 & 32.0 & 29.9 & 33.1 & 25.1 & 47.9 & 28.4 & 25.2 & -- & -- & -- & -- \\
    \hline

    % --- Proprietary Models ---
    \rowcolor{navyblue!5}
    \multicolumn{14}{l}{\textcolor{black}{\textit{Non-Thinking Models}}} \\
    GPT-4o & 10 & 31.9 & 23.9 &  \cellcolor{oai-gray-600}{\textcolor{white}{38.6}} & 29.8 & 24.2 & 36.2 & 31.2 & 43.6 & 29.3 & 40.5 & 25.8 & 39.2 \\ 
    % GPT5 & 4 & 46.7 & 44.5 & \cellcolor{oai-gray-600}{\textcolor{white}{34.0}} & 58.4 & \cellcolor{oai-gray-600}{\textcolor{white}{77.9}} & 36.1 & 60.6 & 55.3 & 38.6 & 52.1 & 23.6 & 19.4 \\
    Gemini2.5 Flash NT & 6 & 35.8 & 31.6 & 29.3 & 30.8 & 35.2 & 45.4 & 36.4 & 53.7 & 28.4 & 36.9 & 27.6 & 45.7 \\
    Gemini2.5 Pro NT & \cellcolor{oai-green-200}{2}  & 41.4 & 36.2 & 30.0 & 33.2 & 47.0 &  \cellcolor{oai-gray-600}{\textcolor{white}{48.5}} & 43.3 & 55.2 & 39.6 & 47.5 &  \cellcolor{oai-gray-600}{\textcolor{white}{29.2}} & 46.0 \\
    
    \hline

    % --- Thinking Models ---
    \rowcolor{navyblue!5}
    \multicolumn{14}{l}{\textcolor{black}{\textit{Thinking Models}}} \\
    % o1-1217 & \cellcolor{oai-green-200}{3} & xx & xx & xx & xx & xx & xx & xx & xx & -- & -- & -- & -- \\
    % o3-high& \cellcolor{oai-green-200}{3} & xx & 5.3 & 43.8 & 38.2 & 37.0 & 41.3 & 31.5 & 28.5 & -- & -- & -- & -- \\
    Seed1.6-Vision  & 7 & 35.7 & 34.0 & 32.0 & 34.9 & 40.0 & 44.4 & 34.5 & 41.5 & 33.3 & 39.2 & 30.1 & 39.0 \\ 
    GLM4.5V & 5  & 36.0 & 35.3 & 24.0 & 32.5 & 34.5 & 43.7 & 34.5 & 34.1 & 33.8 & 41.0 & 26.8 & 49.7 \\ 
    Gemini2.5-Pro & \cellcolor{oai-green-600}{1}  & \cellcolor{oai-gray-600}{\textcolor{white}{50.1}} & \cellcolor{oai-gray-600}{\textcolor{white}{47.8}} & 32.6 & \cellcolor{oai-gray-600}{\textcolor{white}{44.4}} & \cellcolor{oai-gray-600}{\textcolor{white}{61.6}} & 47.9 & \cellcolor{oai-gray-600}{\textcolor{white}{50.5}} & \cellcolor{oai-gray-600}{\textcolor{white}{61.5}} & \cellcolor{oai-gray-600}{\textcolor{white}{47.6}} & \cellcolor{oai-gray-600}{\textcolor{white}{58.2}} & 28.3 &  \cellcolor{oai-gray-600}{\textcolor{white}{63.3}} \\
    Gemini2.5-Flash & 4 & 39.0 & 33.5 & 23.3 & 35.8 & 54.0 & 42.0 & 42.5 & 55.3 & 33.8 & 45.3 & 27.1 & 39.8 \\ 
    % Qwen3VL-30B-thinking & 15  & 36.3 & 29.6 & 28.6 & 39.8 & 38.4 & 41.8 & 47.2 & 25.2 & 42.2 & 42.1 & 14.4 & 42.1 \\
    % Qwen3VL-235B-thinking & 9  & 33.1 & 24.4 & 22.0 & 28.6 & 37.5 & 37.3 & 38.9 & 35.1 & 32.6 & 38.4 & 23.9 & 38.6 \\
    % claude4-sonnet & --  & -- & -- & -- & -- & -- & -- & -- & -- & -- & -- & -- & -- \\ 
    % Kimi-VL-A3B-Thinking & -- & -- & -- & -- & -- & -- & -- & -- & -- & -- & -- & 16.2 & 7.8 \\ 
    \hline

    % --- Open-source Models ---
    \rowcolor{navyblue!5}
    \multicolumn{14}{l}{\textcolor{black}{\textit{Open-source Models}}} \\

    Qwen2.5VL-7B & 12 & 27.5  & 17.8 & 22.6 & 23.9 & 20.6 & 31.6 & 34.7 & 15.2 & 28.4 & 39.8 & 24.5 & 31.1 \\
    Qwen2.5VL-32B & 11  & 27.9 & 21.4 & \cellcolor{oai-gray-300}{30.0} & 25.8 & 22.0 & 35.1 & 29.0 & 21.7 & 32.8 & 37.0 & 14.1 & 38.6 \\
    Qwen2.5VL-72B & 9  & 33.0 & 24.4 & 22.0 & 28.6 & \cellcolor{oai-gray-300}{37.5} & 37.3 & 38.9 & 35.1 & 32.6 & 38.4 & 23.9 & 38.6 \\
    Qwen3VL-30B & 8  & 35.3 & 31.9 & 26.0 & 32.3 & 20.7 & \cellcolor{oai-gray-300}{39.2} & 37.8 & 48.1 & 32.7 & 44.2 & 25.8 & 40.9 \\
    Qwen3VL-235B & \cellcolor{oai-green-400}{3}  & \cellcolor{oai-gray-300}{40.0} & \cellcolor{oai-gray-300}{33.9} & 27.4 & \cellcolor{oai-gray-300}{35.1} & 35.4 & 38.9 & \cellcolor{oai-gray-300}{43.7} & \cellcolor{oai-gray-300}{53.4} & \cellcolor{oai-gray-300}{37.3} & \cellcolor{oai-gray-300}{44.7} & \cellcolor{oai-gray-300}{28.8} & \cellcolor{oai-gray-300}{48.9} \\ 
    Kimi-VL-A3B & 13 & 24.4 & 13.8 & 23.3 & 31.6 & 27.0 & 21.4 & 24.9 & 28.3 & 26.9 & 27.8 & 15.7 & 32.6 \\
    % InternVL3.5-14B & 11  & 25.7 & 24.0 & 20.0 & 29.2 & 32.1 & \cellcolor{oai-gray-300}{44.1} & 30.4 & 6.3 & -- & -- & -- & -- \\
    % InternVL3.5-38B & 11 & 25.7 & 24.0 & 20.0 & 29.2 & 32.1 & \cellcolor{oai-gray-300}{44.1} & 30.4 & 6.3 & -- & -- & -- & -- \\
    \hline
    \end{tabular}
    } % End of resizebox
    \vspace{-0.2cm}
    \caption{\textbf{Our-Bench.} \colorbox{oai-gray-600}{\textcolor{white}{Dark gray}} indicates the best result among all models and \colorbox{oai-gray-300}{light gray} indicates the best result among open-source models. NT denotes the non-thinking model. \textbf{Avg} is computed using the weights in brackets [\textcolor{gray}{$\cdot$}].}
    \label{tab:main_table}
    % \vspace{-0.1em}
\end{table*}

We first present the overall performance of all benchmarked models on our proposed SpatialTree-Bench, with detailed results summarized in Tab.~\ref{tab:main_table}. In our benchmark, the reasoning models achieve clear improvement over their non-thinking versions, e.g., Gemini2.5-Pro (53.9) vs. Gemini2.5-Pro-NT (51.4). 

\section{Exploring Ability Dependencies and Hierarchical Transfer}
\label{sec:ability_influence}

\subsection{Analysis of Ability Dependencies}
\label{sec:ability_dependencies}

\begin{figure}[!t]
    \centering
    \includegraphics[width=0.45\textwidth]{figures/pearson_heatmap.pdf}
    \caption{\textbf{Inter-Capability Dependencies via Pearson Correlation.} 
    (A) Correlation matrix among higher-level capabilities (L3 and L4); 
    (B) Correlation matrix among foundational L1 capabilities; 
    (C) Salient low-level abilities influencing higher-level tasks.}
    \label{fig:wrap_example}
\end{figure}

To investigate the structure of spatial intelligence in MLLMs, we analyze dependencies among fine-grained sub-abilities using Pearson correlation coefficients computed from our benchmark scores. A high positive correlation indicates that strong performance on one ability tends to accompany strong performance on another. Fig.~\ref{fig:wrap_example} presents a heatmap of these correlations across all models. Notably, higher-level capabilities (L3 and L4) exhibit stronger correlations (region A), suggesting that complex tasks such as route planning and causal reasoning rely on overlapping foundational sub-skills. In contrast, lower-level abilities (L1) show weak correlations, indicating they are largely independent. Based on this coarse correlation analysis, we select several underutilized low-level abilities for further investigation. These abilities are explored through Supervised Fine-Tuning (SFT) and explicit prompting to examine their influence and transfer, both within the same level and across higher levels.

% Finally, we identify a set of low-level critical abilities that act as prerequisites for a wide range of higher-level competencies. For example, strong performance in geometric perception tasks, particularly distance estimation (\textit{L1-Geo.Dist}) and size estimation (\textit{L1-Geo.Size}), shows a strong positive correlation with many advanced abilities, including open exploration (\textit{L4-Open.Expl.}), and causal reasoning (\textit{L3-Seq.Plan.Ope, L3-Caus.Reas.Rel}). This indicates that a model's ability to perceive fundamental geometric properties is a cornerstone upon which more abstract spatial reasoning is constructed. These findings strongly support our hypothesis that a core set of atomic abilities forms the basis for the emergence of broader spatial intelligence in MLLM.

\subsection{Probing Cross-Ability Transfer via SFT}
\label{sec:sft_exp}
\begin{table*}[ht!]
    \vspace{-0.4cm}
    \centering
    \resizebox{\linewidth}{!}{
    \begin{tabular}{l|c|ccccc|cc|cc|cc}
    
    \hline % <--- THIS IS THE NEW "CAP" LINE TO CLOSE THE TOP OF THE TABLE
    
    % --- Header Definition with new colors ---
    \rowcolor{white} 
     & & 
    \multicolumn{5}{c|}{\cellcolor{L1_Purple} L1 Perception} &  
    \multicolumn{2}{c|}{\cellcolor{L2_Orange} L2 Mental Mapping} &
    \multicolumn{2}{c|}{\cellcolor{L3_Green} L3 Mental Simulation} &
    \multicolumn{2}{c}{\cellcolor{L4_Blue} L4 Agentic Competence} \\
    
    \rowcolor{white} 
    \textbf{Methods} & \textbf{Avg.} & 
    \textbf{Geom.} & \textbf{Motion} & \textbf{Rel.} & \textbf{Local.} & \textbf{Orient.} &
    \textbf{Underst.} & \textbf{Memory} &
    \textbf{Caus. Reas.} & \textbf{Seq. Plan.} &
    \textbf{Goal Exec.} & \textbf{Open Expl.} \\
    % \hline

    % % --- Baselines ---
    % \rowcolor{navyblue!5}
    % \multicolumn{15}{l}{\textcolor{black}{\textit{Baseline}}} \\
    % Chance Level (Random) & - & - & -- & -- & -- & -- & -- & -- & -- & 25.0 & 36.1 & 28.3 & 25.0 & -- \\
    % Chance Level (Frequency) & - & 34.0 & 62.1 & 32.0 & 29.9 & 33.1 & 25.1 & 47.9 & 28.4 & 25.2 & -- & -- & -- & -- \\
    \hline

    % --- Proprietary Models ---
    \rowcolor{navyblue!5}
    Baseline & 25.0 & 20.9 & 28.6 & 28.9 & 24.2 & 34.2 & 22.6 & 21.7 & 27.2 & 31.7 & 22.1 & 26.5 \\ 
    B+Dist. & 24.5 & \goodres{24.1}{3.2} & \badres{26.6}{-2.0} & \badres{23.2}{-5.8} & \badres{19.6}{-4.6} & \neutralres{34.3}{0.1} & \goodres{24.6}{2.0} & \neutralres{21.8}{0.1} & \neutralres{26.1}{-1.1} & \neutralres{30.8}{-0.9} & \goodres{25.5}{3.4} & \neutralres{26.1}{-0.4} \\ 
    B+Corr. & 25.2 & \badres{17.6}{3.2} & \badres{23.9}{-4.7} & \goodres{30.2}{1.3} & \badres{18.9}{-5.3} & \goodres{35.6}{1.4} & \neutralres{21.9}{-0.7} & \goodres{24.6}{2.9} & \badres{21.8}{-5.4} & \goodres{33.9}{2.2} & \goodres{24.7}{2.6} & \goodres{35.9}{9.4} \\ 
    B+Size & 23.5 & \badres{24.3}{3.4} & \badres{22.6}{-6.0} & \badres{21.4}{-7.5} & \badres{21.7}{-2.5} & \neutralres{34.5}{0.3} & \neutralres{21.9}{-0.8} & \badres{19.2}{-2.5} & \badres{23.4}{-3.8} & \badres{30.3}{-1.5} & \badres{21.5}{-0.6} & \badres{24.3}{2.2} \\ 
    B+Dist.+Size+Corr. & 26.1 & \goodres{25.5}{4.6} & \neutralres{29.3}{0.7} & \neutralres{29.4}{0.5} & \badres{16.4}{-7.8} & \neutralres{33.7}{0.5} & \goodres{23.0}{0.4} & \goodres{24.2}{2.5} & \badres{25.2}{-2.0} & \goodres{34.2}{2.5} & \goodres{26.0}{3.9} & \goodres{28.5}{2.0} \\ 
    B+Dist.+Size+Corr.+Mot. & 27.3 & \goodres{28.6}{7.7} & \badres{24.6}{-4.0} & \badres{20.6}{-8.3} & \goodres{26.3}{2.1} & \goodres{36.0}{1.8} & \neutralres{22.2}{-0.4} & \neutralres{22.6}{0.9} & \goodres{28.2}{1.0} & \goodres{32.8}{1.1} & \goodres{23.3}{1.1} & \goodres{35.9}{9.4} \\ 
    Baseline+50@(all spat.) & 23.6 & \goodres{24.9}{4.0} & \badres{22.6}{-6.0} & \badres{25.9}{-3.0} & \badres{17.4}{-6.8} & \badres{31.2}{-3.0} & \neutralres{22.2}{-0.4} & \badres{20.6}{-1.1} & \badres{25.7}{-1.5} & \badres{30.2}{-1.5} & \badres{19.7}{-2.4} & \badres{22.8}{-3.7} \\ 
    \hline
    \end{tabular}
    } % End of resizebox
    \vspace{-0.2cm}
    \caption{\textbf{SFT Comparisons.}
    "B+Dist.", "B+Corr.", and "B+Size" denote the baseline augmented with distance, correspondence, and size tuning data, respectively. Changes are color-coded as \textcolor{papergreen}{notable gains}, \textcolor{gray}{neutral influence}, and \textcolor{paperred}{drops}.}
    \label{tab:sft_exp}
    \vspace{-1.7em}
\end{table*}


\begin{figure*}[htbp]
\centering
\includegraphics[width=0.98\linewidth]{figures/ability_transfer.pdf}
\vspace{-1em}
\caption{\textbf{Demonstration of Capability Transfer after Distance SFT.} 
\textbf{(Top)} The model is trained on distance QAs, such as object depth sorting and comparison, just using data from synthetic and indoor scenes. 
\textbf{(Middle)} This learned capability transfers in a zero-shot manner to complex reasoning tasks in unseen, in-the-wild scenes, achieving a \textbf{36.0\%} performance gain over the baseline. 
\textbf{(Bottom)} Furthermore, the skill exhibits cross-level transfer, enabling the model to perform a robotic arm manipulation task with a \textbf{27.1\%} performance gain.}
\vspace{-1.5em}
\label{fig:ability_transfer}
\end{figure*}

\begin{tcolorbox}[
    colback=seedblue!5,
    colframe=seedblue!80,
    boxrule=0.5pt,
    arc=2pt,
    left=6pt,
    right=6pt,
    top=6pt,
    bottom=6pt,
    fonttitle=\bfseries,
    title=\textbf{Finding 1}
]
\textbf{Cross-Ability Transfer:} Single-ability L1 SFT induces cross-level transfer, while yielding limited or slightly negative effects on same-level abilities.
\end{tcolorbox}

Based on a naive Pearson correlation analysis, we manually select three L1 abilities that exhibit the strongest correlations with higher-level performance: Geometry Distance (\textit{L1-Geo.Dist}), Geometry Size (\textit{L1-Geo.Size}), and Relative Correlation (\textit{L1-Relat.Corr}). 

\noindent\textbf{General Data Mixture.} To construct the general visual-instruction data, we follow the VST~\cite{vst_yang2025} data mixing recipe and combine multimodal datasets from LLaVA-Video~\citep{zhang2024llavanext-video}, LLaVA-NeXT-Interleave~\citep{li2024llava_next_interleave}, and LLaVA-OneVision~\citep{li2024llava_onevision}, covering single-image, multi-image, video, and 3D tasks. We then use SpatialEngine to generate ability-specific instruction data, mixed with the general data in a 1:3 ratio specifically for each. To isolate the gains from general data, we use a baseline fine-tuned only on the general data with the same token consumption. 

\noindent\textbf{Targeted SFT Data.} For \textit{L1-Geo.Dist}, we generate approximately 0.25M distance-relevant QA samples from SUNRGBD~\citep{song2015sun}, Hypersim~\citep{Hypersim}, and Matterport3D~\citep{Matterport3D}. The training data is further augmented with visual prompts and multi-scale transformations to enhance distance reasoning. For \textit{L1-Relat.Corr}, we generate matching data following VST~\cite{vst_yang2025}, sampling 0.25M examples. Similarly, for \textit{L1-Geo.Size}, we generate 0.25M samples from 3D bounding-box annotated datasets, including SUNRGBD, Hypersim and ArkitScenes~\citep{baruch2021arkitscenes}. 

\noindent\textbf{Results and Analysis.} As shown in Tab.~\ref{fig:ability_transfer}, single-ability SFT on distance, correspondence, or size generally yields negligible gains or even substantial drops in other abilities at the same level. Specifically, \texttt{B+Dist.} increases \textbf{Geom.} abilities by +3.2, while decreasing \textbf{Motion}, \textbf{Rel.}, and \textbf{Local.} by -2.0, -5.8, and -4.6, respectively. However, it provides non-trivial gains in higher-level abilities, notably \textbf{Underst.} (+2.0) and \textbf{Goal Exec.} (+3.4). To give a further exploration on how this cability transfer happens and why, we provide a qualitative examples in Fig.~\ref{fig:ability_transfer}. After being fine-tuned on distance-QA data, \texttt{B+Dist.} can generalize to much more complex distance-related questions in in-the-wild scenarios, including those with novel coordinate prompts and multiple points queried simultaneously. This indicates that the model has learned an awareness of distance rather than overfitting to specific QA templates. Besides, and more intriguingly, the improved distance ability also shows clear cross-level transfer. It benefits higher-level tasks such as robot-arm manipulation, where MLLMs are required to guide the gripper to move, rotate, and open/close in 3D space. A better sense of metric space helps the model generate more reasonable control decisions in the real world.

\begin{tcolorbox}[
    colback=seedblue!5,
    colframe=seedblue!80,
    boxrule=0.5pt,
    arc=2pt,
    left=6pt,
    right=6pt,
    top=6pt,
    bottom=6pt,
    fonttitle=\bfseries,
    title=\textbf{Finding 2}
]
\textbf{Multi-ability Synergy:} 
The holistic integration across multiple fundamental abilities achieves synergistic gains far exceeding their individual effects.
% While L1-Relat.Corr alone yields negligible gains, combining it with L1-Geo.Dist results in substantially larger improvements, exceeding those of L1-Geo.Dist alone.
\end{tcolorbox}
Tab.~\ref{tab:sft_exp} reveals an interesting phenomenon: individual SFT on any single ability—Distance, Size, or Correspondence—has limited impact on overall spatial performance, and can even slightly reduce it (e.g., \texttt{B+Dist.} \textbf{-0.5}, \texttt{B+Corr.} \textbf{+0.2}, \texttt{B+Size.} \textbf{-1.5} relative to the baseline). In contrast, combining all three abilities in a blended SFT (\texttt{B+Dist.+Size+Corr.}) yields an overall gain of \textbf{+1.1}, surpassing the performance of any individual ability and even exceeding the sum of their separate contributions. Remarkably, for abilities that suffered substantial drops under single-ability SFT—such as L1.Motion (best individual change \textbf{-2.0})—the compositional training produces a positive improvement of \textbf{+0.7}.

\subsection{Reinforcement Learning}
\begin{table*}[ht!]
    \vspace{-0.4cm}
    \centering
    \resizebox{\linewidth}{!}{
    \begin{tabular}{l|c|ccccc|cc|cc|cc}
    
    \hline % <--- THIS IS THE NEW "CAP" LINE TO CLOSE THE TOP OF THE TABLE
    
    % --- Header Definition with new colors ---
    \rowcolor{white} 
     & & 
    \multicolumn{5}{c|}{\cellcolor{L1_Purple} L1 Perception} &  
    \multicolumn{2}{c|}{\cellcolor{L2_Orange} L2 Mental Mapping} &
    \multicolumn{2}{c|}{\cellcolor{L3_Green} L3 Mental Simulation} &
    \multicolumn{2}{c}{\cellcolor{L4_Blue} L4 Agentic Competence} \\
    
    \rowcolor{white} 
    \textbf{Methods} & \textbf{Avg.} & 
    \textbf{Geom.} & \textbf{Motion} & \textbf{Rel.} & \textbf{Local.} & \textbf{Orient.} &
    \textbf{Underst.} & \textbf{Memory} &
    \textbf{Caus. Reas.} & \textbf{Seq. Plan.} &
    \textbf{Goal Exec.} & \textbf{Open Expl.} \\
    % \hline

    % % --- Baselines ---
    % \rowcolor{navyblue!5}
    % \multicolumn{15}{l}{\textcolor{black}{\textit{Baseline}}} \\
    % Chance Level (Random) & - & - & -- & -- & -- & -- & -- & -- & -- & 25.0 & 36.1 & 28.3 & 25.0 & -- \\
    % Chance Level (Frequency) & - & 34.0 & 62.1 & 32.0 & 29.9 & 33.1 & 25.1 & 47.9 & 28.4 & 25.2 & -- & -- & -- & -- \\
    \hline

    % --- Proprietary Models ---
    \rowcolor{navyblue!5}
    Qwen2.5-VL-7B & 27.5 & 17.8 & 22.6 & 23.9 & 20.6 & 21.6 & 34.7 & 15.2 & 28.4 & 39.8 & 24.5 & 31.1 \\ 
    After RL & 28.2 & \goodres{19.0}{1.2} & \goodres{25.2}{2.7} & \neutralres{23.6}{-0.4} & \neutralres{20.3}{-0.3} & \neutralres{31.5}{-0.1} & \badres{30.6}{-4.1} & \badres{13.1}{-2.1} & \goodres{33.6}{5.2} & \badres{26.8}{-13.0} & \goodres{29.7}{5.2} & \goodres{40.0}{9.0} \\ 

    \hline
    \end{tabular}
    } % End of resizebox
    \caption{\textbf{RLVR Comparisons.}
    The table compares the baseline Qwen2.5-VL-7B with the version enhanced by RL on Goal-Exec. tasks. Changes are color-coded as \textcolor{papergreen}{notable gains}, \textcolor{gray}{neutral influence}, and \textcolor{paperred}{drops}.}
    \label{tab:rl_exp}
\end{table*}
\label{sec:RL}
In the main paper, we demonstrated the efficacy of Supervised Fine-Tuning (SFT) and prompting in activating spatial capabilities. Motivated by recent advancements in reasoning models (e.g., DeepSeek-R1), we further investigate whether Reinforcement Learning with Verifiable Rewards (RLVR) can serve as a catalyst for emerging spatial intelligence.

\noindent\textbf{Methodology.} We employ Group Relative Policy Optimization (GRPO) to train the Qwen2.5-VL-7B model. To construct a rewardable environment, we utilize robotic arm manipulation data (distinct from the evaluation set) and reformat the perception and planning tasks into Multiple-Choice Questions (MCQs). This format allows for deterministic, rule-based verification of the model's outputs (i.e., correct/incorrect selection) to serve as the reward signal during RL training.

\noindent\textbf{Strict Evaluation Setup.} It is important to note two critical factors in our experimental design that ensure the robustness of our findings:
\begin{itemize}
    \item Data Decontamination: The robotic arm data samples used for GRPO training are strictly separated from the SpatialTree-Bench testing data. There is no overlap in specific scenes or object configurations between the training and evaluation sets.
    \item Task and Metric Discrepancy: The training objective is purely maximizing the reward on discrete MCQ selection. In contrast, the SpatialTree-Bench evaluation employs a diverse set of continuous and semantic metrics (e.g., Mean Squared Error for distance, angular error for orientation, and execution success rates for agentic tasks).
\end{itemize}

\noindent\textbf{Results and Observations.} Despite the significant domain gap and the difference in task formulation, the GRPO-tuned Qwen2.5-VL-7B demonstrates notable improvements across multiple levels of the SpatialTree hierarchy compared to its base counterpart. This suggests that the model is not merely memorizing dataset-specific patterns, but is effectively internalizing generalized spatial reasoning policies through the reinforcement learning process. These preliminary findings highlight the potential of RLVR as a scalable pathway for advancing spatial intelligence in MLLMs.

\subsection{Ability Transfer via Prompting}
\label{sec:atomic_prompting}

\begin{figure}[htbp]
\centering
\includegraphics[width=1\linewidth]{figures/visual_corr_compare.pdf}
\vspace{-1em}
\caption{\textbf{Correspondence Prompting for Navigation.} The correspondence prompt guides Gemini2.5-pro to navigate and move  more accurately within 3D environments.}%
\vspace{-1em}
\label{fig:prompt}
\end{figure}

In addition to SFT, we investigate cross-level ability influence through explicit prompting. Specifically, we consider a representative task pair: low-level abilities (L1.Corr, L1.Dist, L1.Size) and a high-level task (L4.Imaged Goal Navigation). Intuitively, correspondence is a necessary component for navigation. Using Gemini2.5-pro, we provide models with explicit prompts derived from matching visualizations, depth, and size context. As shown in Fig.~\ref{fig:prompt}, correspondence guidance improves target direction recognition, increasing accuracy by \textbf{7.1\%}, while distance and size prompting yield gains of \textbf{5.5\%} and \textbf{2.1\%}, respectively. These results suggest that grounding MLLM reasoning with explicit low-level visual information can substantially enhance performance on complex spatial navigation tasks.
% 
% Specifically, we compared a baseline condition, where the model generated actions from inital and final states given a defined 6DoF action space, with an experimental condition that augmented the model input with explicit visual correspondence figures. 


\vspace{-1em}
\section{Conclusion and Future works}
We present SpatialTree, the first capability-centric framework for Spatial Intelligence, organizing abilities into four hierarchical layers. This structure enables analysis of how spatial abilities emerge, compose, and transfer across levels. It also opens opportunities to efficiently scale up spatial intelligence in MLLMs, by strategically leveraging different types of data: identifying which abilities are most effective for pre-training, which can be directly applied in reinforcement learning with minimal additional reasoning data during post-training, and which are acquired through real-world interactions. We believe this could provide a promising path toward advancing spatial intelligence in MLLMs.







\clearpage

\bibliographystyle{plainnat}
\bibliography{main}

\clearpage

\beginappendix

% Setting appendix numbering style
% Note: \appendix is already called by \beginappendix in paper.tex
\renewcommand\thefigure{\Alph{figure}} % redefining the figure numbering style
\renewcommand\thetable{\Alph{table}}   % redefining the table numbering style
\renewcommand\thesection{\Alph{section}}   % redefining the section numbering style
\setcounter{figure}{0} % reset counter 
\setcounter{table}{0} % reset counter
\setcounter{section}{0} % reset counter

% Note: Appendix title is already created by \beginappendix command
\label{sec:appendix}

% Table of contents for appendix
\tableofcontents

\newpage

\section{Visualization of Data Sources}
How different datasets contribute to our SpatialTree evaluation is shown in Fig.~\ref{fig:datasource}.
\begin{figure}[t]
    \centering
    \includegraphics[width=0.98\linewidth]{figures/datasource_new.png}
   \caption{\textbf{Construction of SpatialTree-Bench.} We build our benchmark by reorganizing various existing datasets and mapping them to our capability tree, where \textbf{SpatialPlus}, a complementary dataset are introduced to ensure the capability coverage.}
    \label{fig:datasource}
\end{figure}


\section{Evaluation Metrics Details}

\paragraph{Multi-Option QAs.} 
For multi-option question answering, each model is evaluated on its ability to select the correct option from a predefined set. We measure accuracy by comparing the predicted choice against the ground-truth answer. This paradigm captures a model's understanding of spatial relations, object properties, and causal dynamics within a scene, corresponding to the low- and mid-level capabilities in the SpatialTree (L1–L3).

\paragraph{Numeric QAs.} 
Numeric QAs require models to predict continuous quantities such as distances, angles, or 3D coordinates. We evaluate performance using relative error metrics, for example: 
\[
\text{Relative Error} = \frac{|\hat{y}-y|}{|y|},
\] 
where $\hat{y}$ is the predicted value and $y$ is the ground truth. This metric ensures that predictions are scaled appropriately across different magnitudes and emphasizes precision in spatial reasoning.

\paragraph{GPT Judge.} 
For tasks that are open-ended or involve complex reasoning (e.g., trajectory description, action sequence explanation), we leverage a GPT-based judge to assess correctness. The judge evaluates whether the generated response satisfies the task requirements, optionally scoring partial correctness. This approach allows flexible evaluation beyond rigid numeric or multiple-choice formats, especially for mid- and high-level capabilities in L3–L4.

\paragraph{Agentic Evaluation.} 
To assess agentic competence, models are deployed in interactive simulated environments, such as those provided by EmbodiedBench~\citep{yang2025embodiedbench}. We evaluate navigation and manipulation tasks along multiple dimensions: success rate in completing the target goal, relative translation accuracy, and directional alignment. For each action step, a combined metric is computed using relative distance and cosine similarity of movement vectors, producing a normalized score in $[0,1]$. Aggregating scores over all steps yields a comprehensive measure of an agent’s ability to plan and execute actions in long-horizon, embodied tasks.


\section{SpatialPlus: Complementary Data Annotations for SpatialTree}

\begin{figure}[t]
    \centering
    \includegraphics[width=0.98\linewidth]{figures/orientation.pdf}
    \caption{\textbf{Orientation Annotations.} The left side is the gravity field estimated from GeoCalib~\cite{veicht2024geocalib}, while the right side is from OrientAnything.}
    \label{fig:orientation}
\end{figure}

\subsection{Orientations (L1)}
The Orientation capability, a fundamental yet under-explored area, involves estimating both gravity direction and 3D object orientation. To generate annotations, we leveraged Geocalib~\cite{veicht2024geocalib} for gravity vector estimation and OrientAnything~\cite{orient_anything} for object poses. We applied these tools to datasets suited for each task: for gravity, we annotated 500 images sampled from the diverse, drone-captured TartanAir~\cite{wang2020tartanair} dataset; for object orientation, we utilized the object-centric Co3dv2~\cite{reizenstein21co3d} dataset (Seen in Fig.~\ref{fig:orientation}). 
For gravity, the goal is to estimate the camera’s orientation relative to the gravity vector, typically represented by the \textit{pitch} and \textit{roll} angles. Formally, let the gravity vector in the world frame be:
\begin{equation}
\mathbf{g}_w = 
\begin{bmatrix}
0 \\ 0 \\ -1
\end{bmatrix},
\end{equation}
and let $\mathbf{R}_{cw} \in SO(3)$ denote the rotation from the world frame to the camera frame. The gravity direction in the camera frame is then:
\begin{equation}
\mathbf{g}_c = \mathbf{R}_{cw} \, \mathbf{g}_w.
\end{equation}
From $\mathbf{g}_c = [g_x, g_y, g_z]^\top$, the pitch and roll angles can be computed as:
\begin{align}
\text{pitch} &= \arctan2(-g_x, \sqrt{g_y^2 + g_z^2}), \\
\text{roll}  &= \arctan2(g_y, g_z).
\end{align}
Here, \textit{pitch} measures the forward--backward tilt of the camera, while \textit{roll} measures the sideways tilt. To evaluate an MLLM's proficiency in this task, we require the model to analyze the input image and return these same three parameters in a structured JSON format. An example of our prompt template is shown in Listing~\ref{lst:orientation_prompt}.
\begin{listing}[H]
\begin{minted}[
    frame=single,
    framesep=2mm,
    baselinestretch=1.2,
    fontsize=\small,
    linenos,
    breaklines=true,
    bgcolor=gray!5,
    xleftmargin=10pt,
    xrightmargin=10pt
]{json}
{
  "role": "system",
  "content": "You are a vision model specialized in estimating camera orientation from images. 
  Your task is to infer the gravity direction from the input image by predicting the 
  camera's pitch and roll angles, as well as the vertical field of view (vFOV).
  Always output your prediction strictly in the following JSON format: 
  {
    \"pitch\": <float, camera pitch angle in degrees>,
    \"roll\": <float, camera roll angle in degrees>,
    \"vfov\": <float, vertical field of view in degrees>
  }  
  Do not include any additional text or explanation outside of the JSON object."
}
\end{minted}
\caption{\textbf{Prompt template} for Orientation Estimation.}
\label{lst:orientation_prompt}
\end{listing}
For evaluation, we move beyond a simple absolute error metric and adopt a probabilistic approach that accounts for the inherent uncertainty of the ground-truth annotations provided by \textit{Geocalib}. For each predicted parameter (pitch, roll, and vFOV), \textit{Geocalib} also outputs an uncertainty value, which we interpret as the standard deviation ($\sigma_{gt}$). We then calculate a normalized similarity score ($S$) for each parameter using a Gaussian kernel, defined as:
\begin{equation}
\label{eq:uncertainty_score}
S(y_{\text{pred}}, y_{\text{gt}}, \sigma_{\text{gt}}) = \exp\left(-\frac{(y_{\text{pred}} - y_{\text{gt}})^2}{2\sigma_{\text{gt}}^2}\right)
\end{equation}
where $y_{\text{pred}}$ is the MLLM's prediction, $y_{\text{gt}}$ is the ground-truth value from \textit{Geocalib}, and $\sigma_{\text{gt}}$ is its associated uncertainty. This scoring function gracefully penalizes deviations from the ground truth: the score is 1 for a perfect match and decays towards 0 as the error increases. Crucially, a larger uncertainty $\sigma_{\text{gt}}$ in the ground truth leads to a slower decay, making the scoring more lenient when the ground truth itself is less certain. The final score for the task is the average of the individual scores for pitch, roll, and vFOV. For object orientation estimation, most of metrics are similar to gravity, and the evaluation are conducted on Azimuth, Polar and Rotation these three angles.

\subsection{Agentic Competence (L4)}
\subsection{Goal-driven Navigation}
\begin{figure}[htbp]
    \centering
    \begin{subfigure}{0.365\linewidth}
        \centering
        \includegraphics[width=\linewidth]{figures/img_pair.pdf}
        \caption{}
        \label{fig:sub_a}
    \end{subfigure}
    \hfill % 子图间填充空白
    \begin{subfigure}{0.6\linewidth}
        \centering
        \includegraphics[width=\linewidth]{figures/navigation.pdf}
        \caption{}
        \label{fig:sub_b}
    \end{subfigure}
    \caption{\textbf{Navigation Data Curation.} 
    (a) shows paired images used for evaluation, where MLLMs are expected to move from left to right. 
    (b) illustrates our curation process: reconstructing metric 3D models and camera trajectories, then converting them into actions.}
    \label{fig:navigation}
\end{figure}

\textbf{Goal-driven Navigation.}
We leverage our SpatialEngine to get the action annotations as shown in Fig.~\ref{fig:navigation}. We first extract the metric pose trajectories from the games videos, and convert them into discrete actions with our spatial action mapping, and then we randomly sample several image pairs from the video with the correspondence checking. For evaluation, the goal is a image, and the MLLMs are supposed to control the character to move to the target positions. We use the prompt template as below:
\begin{listing}[H]
\begin{minted}[
    frame=single,
    framesep=2mm,
    baselinestretch=1.2,
    fontsize=\small,
    breaklines=true,
    bgcolor=gray!5,
    xleftmargin=10pt,
    xrightmargin=10pt
]{json}
{
  "role": "system",
  "content": "Task Details:\n
Analyze Images: Compare the start image <Image 1> and the target image <Image 2> to understand the required translation and rotation for the robot arm's end-effector.\n
Define Motion: Decompose the movement into 6 steps, each containing one or more elementary actions.\n
Quantify Actions: For each action, specify an integer step_num that represents its intensity.\n\n
Coordinate System:\n
Right-hand frame attached to the end-effector: +Z forward, +X right, +Y downward.\n\n
Action Space:\n
Translation: Dolly In (W), Dolly Out (S), Truck Left (A), Truck Right (D), Pedestal Up (space), Pedestal Down (shift).\n
Rotation: Pan Left (left arrow), Pan Right (right arrow), Tilt Up (up arrow), Tilt Down (down arrow), Roll CW (clockwise), Roll CCW (counterclockwise).\n
Special Action: Stay (STOP).\n\n
Step Size:\n
Translation: 0.019375 m/step. Rotation: 0.4509 rad/step.\n\n
Output Format:\n
Return a single JSON object with keys step_1–step_6. Each step contains:\n
  actions: list of action symbols\n
  step_nums: corresponding integers.\n\n
Example:\n
{
  \"step_1\": {
    \"actions\": [\"W\", \"A\"],
    \"step_nums\": [5, 2]
  },
  \"step_2\": {
    \"actions\": [\"W\", \"up_arrow\"],
    \"step_nums\": [3, 4]
  }
}"
}
\end{minted}
\caption{\textbf{Prompt of navigation.}}
\label{lst:navigation_p}
\end{listing}
In this prompt, translation and rotation steps are computed from the actual movement, while capping the number of steps at 10 to prevent overly long action sequences. To evaluate MLLMs, we compute a normalized metric in the range $[0, 1]$ by combining \textbf{relative distance} and \textbf{directional accuracy}. Specifically, for each step, let $\Delta \mathbf{p}_{\text{pred}}$ and $\Delta \mathbf{p}_{\text{gt}}$ denote the predicted and ground-truth translation vectors, respectively. 

The \textbf{relative distance score} is defined as:
\[
s_d = \max\Big(0, 1 - \frac{\|\Delta \mathbf{p}_{\text{pred}} - \Delta \mathbf{p}_{\text{gt}}\|}{\|\Delta \mathbf{p}_{\text{gt}}\|}\Big),
\]

and the \textbf{directional score} is computed by the cosine similarity:
\[
s_\theta = \frac{\Delta \mathbf{p}_{\text{pred}} \cdot \Delta \mathbf{p}_{\text{gt}}}{\|\Delta \mathbf{p}_{\text{pred}}\| \, \|\Delta \mathbf{p}_{\text{gt}}\|}.
\]

The final step-wise accuracy is then: $s_{\text{step}} = s_d \cdot \max(0, s_\theta)$

which ensures a value in $[0,1]$, where 1 indicates perfect alignment in both distance and direction. Aggregating $s_{\text{step}}$ across all steps provides a comprehensive measure of the model's precision in executing end-effector motions.

\textbf{Goal-driven Manipulation}
For the \textbf{Goal-Driven Manipulation} capability, we utilize action annotations from the \texttt{Droid}~\cite{khazatsky2024droid} and \texttt{EgoDex}~\cite{egodex} datasets. This task requires the MLLM to generate a sequence of precise actions to move a robot end-effector or a human hand from a starting state to a target state, both specified by images. The action space for \texttt{Droid} encompasses 7-DoF control: 6-DoF for the end-effector's pose (translation and rotation) and a binary state for the gripper (open/close). A similar action space is adapted for \texttt{EgoDex}, controlling wrist pose and finger grip. The MLLM is prompted to generate a sequence of continuous action vectors, as shown in the template below:

\begin{listing}[t]
\begin{minted}[
    frame=single,
    framesep=2mm,
    baselinestretch=1.2,
    fontsize=\small,
    breaklines=true,
    bgcolor=gray!5,
    xleftmargin=10pt,
    xrightmargin=10pt
]{json}
{
  "role": "system",
  "content": "Task Details:\n
Compare the start image <Image 1> and target image <Image 2> to infer the translation and rotation required for the robot arm's end-effector.\n
Decompose the motion into up to 6 steps, each combining any number of elementary actions.\n\n
Action Space:\n
We define a 7D action vector per step:\n
[dx, dy, dz, d_roll, d_pitch, d_yaw, gripper_state]\n
- Translation (dx, dy, dz): Displacement in meters along +X, +Y, +Z.\n
- Rotation (d_roll, d_pitch, d_yaw): Rotation in radians about +Z, +X, +Y respectively.\n
- gripper_state: 0=open, 1=closed.\n\n
Each dx, dy, dz, d_roll, d_pitch, d_yaw is computed from selected actions and their step_nums:\n
delta_q = step_num * unit_step_size  (translation in meters or rotation in radians)\n\n
Available Actions:\n
W/S: Dolly In/Out (+/-Z)\n
A/D: Truck Left/Right (-/+X)\n
space/shift: Pedestal Up/Down (-/+Y)\n
left_arrow/right_arrow: Pan Left/Right (+/- yaw)\n
up_arrow/down_arrow: Tilt Up/Down (+/- pitch)\n
clockwise/counterclockwise: Roll CW/CCW (+/- roll)\n
STOP: No movement\n\n
Output Format:\n
Return a single JSON object where each step is a key (\"step_1\", \"step_2\", ...).\n
Each step contains:\n
- actions: a list of action symbols\n
- step_nums: a list of integers specifying intensity (1–10)\n
- gripper: 0 or 1 for gripper state\n\n
Example:\n
{
  \"step_1\": {
    \"actions\": [\"W\", \"A\"],
    \"step_nums\": [5, 2],
    \"gripper\": 0
  },
  \"step_2\": {
    \"actions\": [\"clockwise\"],
    \"step_nums\": [3],
    \"gripper\": 1
  }
}"
}
\end{minted}
\caption{\textbf{Prompt for Goal-Driven Manipulation with 7D Action Representation.}}
\label{lst:manipulation_p}
\end{listing}

To evaluate the MLLM's performance, we assess the accuracy of the predicted action sequence against the ground truth. For the translational component of the motion, we reuse the step-wise accuracy metric $s_{\text{step}}$ from the navigation task, which combines relative distance and directional scores. For the rotational component, we compute a normalized score based on the angular difference between the predicted orientation and the ground truth. Let $R_{\text{pred}}$ and $R_{\text{gt}}$ be the predicted and ground-truth rotation matrices for a step. The rotational error angle $\theta_{\text{err}}$ is calculated from the error rotation matrix $R_{\text{err}} = R_{\text{pred}} R_{\text{gt}}^T$:
\[
\theta_{\text{err}} = \arccos\left(\frac{\text{Tr}(R_{\text{err}}) - 1}{2}\right).
\]
The \textbf{rotation score} $s_{\text{rot}}$ is then defined as:
\[
s_{\text{rot}} = \max\Big(0, 1 - \frac{\theta_{\text{err}}}{\pi}\Big),
\]
which normalizes the error to a $[0, 1]$ range, where 1 indicates a perfect rotational match. Finally, the \textbf{gripper score} $s_{\text{gripper}}$ is a binary accuracy (1 if the predicted state matches the ground truth, 0 otherwise). The final score for each step is a weighted combination of these three metrics, providing a holistic evaluation of the model's ability to perform precise, multi-faceted manipulation tasks.

\section{Embodied Agent Evaluation within Simulation}

EmbodiedBench~\citep{yang2025embodiedbench} provides a closed-loop evaluation framework in which MLLMs are deployed within interactive simulators. It includes four primary environments—\textit{EB-ALFRED}, \textit{EB-Habitat}, \textit{EB-Navigation}, and \textit{EB-Manipulation}—supporting long-horizon tasks that require both high-level planning and low-level control. Following the benchmark’s evaluation protocol, we assess our models’ navigation and manipulation capabilities in these simulated settings.

\section{Benchmark Metric Aggregation}

To derive a single, comprehensive score for a model's spatial intelligence, we employ a hierarchical aggregation methodology. This approach is designed to reflect the complex, multi-layered nature of spatial cognition, rather than treating all abilities as equally important. The design is principally guided by established theories in cognitive psychology, which posit that spatial intelligence is constructed hierarchically, with fundamental perceptual skills forming the bedrock for more abstract reasoning and planning.

Our aggregation framework is built upon the \textit{SpatialTree} structure. The assignment of weights within this tree is determined by a synthesis of theoretical principles and empirical, data-driven insights:

\begin{figure}[htbp]
\centering
\includegraphics[width=0.98\linewidth]{figures/weighted_tree.pdf}
\caption{An illustration of the hierarchical weighting scheme for metric aggregation with in the \textit{SpatialTree}. Each node represents a capability layer, with the assigned weight used for the bottom-up calculation of the final score. The weighting prioritizes foundational perceptual abilities (L1) as they are prerequisites for higher-level cognitive tasks.}%
\label{fig:weighted_tree}
\end{figure}

\textbf{Cognitive Hierarchy.} In line with cognitive science literature, our weighting scheme prioritizes foundational capabilities, as shown in Fig.~\ref{fig:weighted_tree}. The L1 layer, which represents low-level spatial perception, is assigned the largest weight, as these skills are prerequisites for almost all higher-level spatial tasks found in L2 (Mental Mapping) and L3 (Mental Simulation).

\textbf{Empirical Dependency from Correlation Analysis.} The theoretical hierarchy is further refined and validated by our empirical findings from the Pearson correlation heatmap (Fig.~\ref{fig:wrap_example}). The heatmap allows us to identify \textit{atomic abilities} that exhibit strong, widespread correlations with a multitude of other skills. These influential abilities are considered more fundamental to the overall spatial intelligence network and are consequently assigned higher weights within their respective sub-trees. This ensures our metric is not just theoretically sound, but also reflects the actual dependencies observed in model performance.

The final score is calculated via a bottom-up, weighted summation. The performance score for any parent node in the tree is the weighted sum of the scores of its immediate children. This process is recursively applied until the root node is reached, yielding a single, principled score that holistically quantifies the spatial intelligence of a given MLLM.



\section{LLM Usage Declarations}
We declare that Large Language Models (LLMs) were used in a limited capacity during the preparation of this manuscript. Specifically, LLMs were employed for grammar checking, word choice refinement, and typo correction. All core technical contributions, experimental design, analysis, and conclusions are entirely our own. The use of LLMs did not influence the scientific methodology, result interpretation, or theoretical contributions of this research.
>>>>>>> 079878c (appendix)


\end{document}